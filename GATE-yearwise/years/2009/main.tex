\iffalse
\chapter{2009}
\author{AI24BTECH11032}
\section{ae}
\fi

\item The linear system of equation Ax=b where $\textbf{A}=\begin{bmatrix}
1 & 2 \\
2 & 4
\end{bmatrix}
$ and $\textbf{b}=\cbrak{\begin{array}{c}
3 \\
3
\end{array}}
$ has
\begin{multicols}{2}
    \begin{enumerate}
        \item no solution
        \item infinitely many solution
        \item a unique solution $x=\cbrak{ \begin{array}{c}
                 1 \\1\end{array}}$
        \item a unique solution $x = \cbrak{ \begin{array}{c}0.5 \\0.5\end{array}}$         
    \end{enumerate}
\end{multicols}
\bigskip
\item The correct iterative scheme for finding the square root of a positive real number R using the Newton Raphson method is
\begin{multicols}{2}
    \begin{enumerate}
        \item $x_{n+1}=\sqrt{R}$
        \item $x_{n+1}=\frac{1}{2}\brak{x_{n}+\frac{R}{x_{n}}}$
        \item $x_{n+1}=\frac{1}{2}\brak{\sqrt{x_{n}}+\sqrt{\brak{x_{n-1}}}}$
        \item $x_{n+1}=\frac{1}{2}\brak{\sqrt{R}+x_{n}}$        
    \end{enumerate}
\end{multicols}
\bigskip
$$\textbf{Common Data Question}$$
$\textbf{Common Data for Question 51 and 52:}$\\
The roots of the characteristics equation for the longitudinal dynamics of a certain aircraft are :
$\lambda_{1}=-0.02+0.2i;\lambda_{2}=-0.02-0.2i; \lambda_{3}=-2.5+2.6i; \lambda_{4}=-2.5-2.6i,$ where $i=\sqrt{-1}.$\bigskip
\item The pair of eigenvalues the represent the phugoid mode is 
\begin{multicols}{4}
    \begin{enumerate}
        \item $\lambda_{1} \textbf{and} \lambda_{3}$
        \item $\lambda_{2} \textbf{and} \lambda_{4}$
        \item $\lambda_{3} \textbf{and} \lambda_{4}$
        \item $\lambda_{1} \textbf{and} \lambda_{2}$        
    \end{enumerate}
\end{multicols}
\bigskip
\item  The short period damped frequency is 
\begin{multicols}{4}
    \begin{enumerate}
        \item $2.6\frac{\textbf{rad}}{\textbf{s}}$
        \item $0.2\frac{\textbf{rad}}{\textbf{s}}$
        \item $2.5\frac{\textbf{rad}}{\textbf{s}}$
        \item $0.02\frac{\textbf{rad}}{\textbf{s}}$        
    \end{enumerate}
\end{multicols}
\bigskip
$\textbf{Common Data for Question 53 and 54 :}$\\
Consider the vector $\overrightarrow{A}=\brak{y^{3}+z^{3}}\hat{i}+\brak{x^{3}+z^{3}}\hat{j}+\brak{x^{3}+y^{3}}\hat{k}$ defined over the unit sphere $x^{2}+y^{2}+z^{2}=1.$\bigskip
\item The surface integral $\brak{\text{taken over the unit sphere}}$ of the component of $\overrightarrow{A}$ normal to the surface is
\begin{multicols}{4}
    \begin{enumerate}
        \item $\pi$
        \item $1$
        \item $0$
        \item $4\pi$        
    \end{enumerate}
\end{multicols}
\bigskip
\item The magnitude of the component of $\overrightarrow{A}$ normal to the spherical surface at the point $\brak{\frac{1}{\sqrt{3}},\frac{1}{\sqrt{3}},\frac{1}{\sqrt{3}}}$ is 
\begin{multicols}{4}
    \begin{enumerate}
        \item $\frac{1}{3}$
        \item $\frac{2}{3}$
        \item $\frac{3}{3}$
        \item $\frac{4}{3}$        
    \end{enumerate}
\end{multicols}
\bigskip
$\textbf{Common Data for Question 55 and 56:}$\\
The partial differential equation for the torsional vibration of a shaft of length L, torsional rigidity GJ, and mass polar moment of inertia per unit length I,is $\text{I}\frac{\partial^{2}\theta}{\partial t^{2}}=\text{GJ}\frac{\partial^{2}\theta}{\partial x^{2}},$ where $\theta$ is the twist. \bigskip
\item If the shaft is fixed at both ends the boundary conditions are:
\begin{multicols}{2}
    \begin{enumerate}
        \item $\frac{\partial \theta}{\partial x}\bigg|_{x=0}=0 \text{ and } \frac{\partial \theta}{\partial x} \bigg|_{x=L}=0$
		
        \item $\theta\brak{0}=0 \text{ and } \theta\brak{L}=0$
        \item $\frac{\partial \theta}{\partial x} \bigg|_{x=0}=0 \text{ and } \theta\brak{L}=0$
        \item $\theta\brak{0}=0 \text{ and } \frac{\partial \theta}{\partial x} \bigg|_{x=L}=0$        
    \end{enumerate}
\end{multicols}    
\bigskip
\item If the $n^{th}$ mode shape of torsional vibration of the above shaft is $\sin\brak{\frac{n\pi x}{L}}$ then the $n^{th}$ natural frequency of vibration ,i.e., $\omega_{n},$ is given by 
\begin{multicols}{2}
    \begin{enumerate}
        \item $\omega_{n}=\frac{n\pi}{L}\sqrt{\frac{GJ}{I}}$
        \item $\omega_{n}=\frac{\brak{2n+1}\pi}{2L}\sqrt{\frac{GJ}{I}}$
        \item $\omega_{n}=\frac{n\pi}{2L}\sqrt{\frac{GJ}{I}}$
        \item $\omega_{n}=\frac{\brak{2n+1}\pi}{L}\sqrt{\frac{GJ}{I}}$        
    \end{enumerate}
\end{multicols}    
\bigskip
$$\textbf{Linked Answer Question}$$
$\textbf{Statement for Linked Answer Question 57 and 58:}$\\
Air enters the combustor of a gas-turbine engine at a total temperature $T_{0}$ of 500K. The air stream is split into two parts: primary and secondary streams. The primary stream reacts with fuel supplied at a fuel-air ratio of 0.05. The resulting combustion products are then mixed with the secondary air stream to obtain gas with total temperature of 1550 K at the turbine inlet. The fuel has a heating value of $42 \frac{MJ}{Kg}.$ The specific heats of air and combustion products are taken $c_{p}=1kj/kg/k.$\bigskip
\item If the sensible enthalpy of fuel is neglected, the temperature of combustion products from the reaction of primary air stream with fuel is approximately
\begin{multicols}{4}
    \begin{enumerate}
        \item $2100\text{K}$
        \item $3200\text{K}$
        \item $2600\text{k}$
        \item $1800\text{K}$        
    \end{enumerate}
\end{multicols}    
\bigskip
\item The approximate ratio of mass flow rates of the primary air stream to the secondary air streams required to achieve the turbine inlet total temperature of $1550\text{K}$ is 
\begin{multicols}{4}
    \begin{enumerate}
        \item $2:1$
        \item $1:2$
        \item $1:1.5$
        \item $1:1$        
    \end{enumerate}
\end{multicols}    
\bigskip
$\textbf{Statement for Linked Answer Question 59 and 60:}$\\
A piston compresses 1 kg of air inside a cylinder as shown. 
\begin{figure}[H]
\centering
\resizebox{0.5\textwidth}{!}{%
\begin{circuitikz}
\tikzstyle{every node}=[font=\LARGE]
\draw [line width=0.9pt, short] (-5,13.75) -- (7.5,13.75);
\draw [line width=0.9pt, short] (-5,13.75) -- (-5,8.75);
\draw [line width=0.9pt, short] (-5,8.75) -- (7.5,8.75);
\draw [line width=0.9pt, short] (1.25,13.75) -- (1.25,8.75);
\draw [line width=0.9pt, short] (1.75,13.75) -- (1.75,8.75);
\draw [line width=0.9pt, short] (1.75,11.25) -- (4.25,11.25);
\draw [line width=0.9pt, ->, >=Stealth] (3.75,12.25) -- (2.5,12.25);
\node [font=\LARGE] at (-2.25,10.75) {Air};
\draw [line width=0.9pt, short] (4.25,11.25) -- (10.75,11.25);
\end{circuitikz}
}%

\label{fig:my_label}
\end{figure}
\item The rate at which the piston does work on the air is $3000\text{W}.$ At the same time, heat is being lost through the walls of the cylinder at a rate of $847.5\text{W}.$
\begin{multicols}{4}
    \begin{enumerate}
        \item $21,525\frac{\textbf{J}}{\textbf{kg}}$
        \item $-21,525\frac{\textbf{J}}{\textbf{kg}}$
        \item $30,000\frac{\textbf{J}}{\textbf{kg}}$
        \item $-8,475\frac{\textbf{j}}{\textbf{kg}}$        
    \end{enumerate}
\end{multicols}
\bigskip
\item Given that the specific heats of air at constant pressure and volume are $c_{p}=1004.5\frac{\text{J}}{\text{kg-K}}$ and $c_{v}=717.7\frac{\text{J}}{\text{kg-K}}$ respectively,the corresponding  changing in the temperature of the air is 
\begin{multicols}{4}
    \begin{enumerate}
        \item $21.4\text{K}$
        \item $-21.4\text{K}$
        \item $30\text{K}$
        \item $-30\text{K}$        
    \end{enumerate}
\end{multicols}



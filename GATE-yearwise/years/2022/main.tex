\iffalse
\chapter{2022}
\author{AI24BTECH11032}
\section{st}
\fi
\item Let $X_{i} = 1, 2,\cdots, n$, be $i.i.d.$ random variables from a normal distribution with
mean 1 and variance 4. Let $S_{n}=X_{1}^{2}+X_{2}^{2}+\cdots+X_{n}^{2}$
. If V$ar\brak{S_{n}}$ denotes the
variance of $S_{n}$, then the value of 
\begin{align*}
    \lim_{n\to\infty}\brak{\frac{Var\brak{S_{n}}}{n}-\brak{\frac{E\brak{S_{n}}}{n}}^{2}}
\end{align*}
$\brak{\text{ in integer }}$ is equal to $\underline{\hspace{2cm}}.$
\bigskip
\item At a telephone exchange, telephone calls arrive independently at an average rate of 1 call per minute, and the number of telephone calls follows a Poisson distribution.
Five time intervals, each of duration 2 minutes, are chosen at random. Let p denote the probability that in each of the five time intervals at most 1 call arrives at the telephone exchange. Then $e^{10}p\brak{\text{ in integer }}$ is equal to $\underline{\hspace{2cm}}.$
\bigskip
\item Let X be a random variable with the probability density function 
\begin{align*}
    f\brak{x} = 
\begin{cases} 
c \brak{x - \sbrak{x}}, & 0 < x < 3, \\
0, & \text{elsewhere}.
\end{cases}
\end{align*}
where c is a constant and $\sbrak{x}$ denotes the greatest integer less than or equal to x. If $A=\sbrak{\frac{1}{2},2}$ then P$\brak{x\in A} \brak{\text{ rounded off to two decimal places }}$ is equal to $\underline{\hspace{2cm}}.$
\bigskip
\item Let X and Y be two random variables such that the moment generating function of X is M$\brak{t}$ and the moment generating function of Y is
\begin{align*}
    H\brak{t}=\brak{\frac{3}{4}e^{2t}+\frac{1}{4}}M\brak{t},
\end{align*}
where $t\in\brak{-h,h},h>0.$ If the mean and the variance of  X are $\frac{1}{2} \text{ and }\frac{1}{4}$ respectively, then the variance of Y $\brak{\text{ in integer }}$ is equal to $\underline{\hspace{2cm}}.$
\bigskip
\item Let $X_{i}=1,2,\cdots n,$ be i.i.d random variables with the probability density function 
\begin{align*}
    f_{X}\brak{x} = 
\begin{cases} 
\frac{1}{\sqrt{2}T\brak{\frac{1}{6}}}x^{-\frac{5}{6}}e^{-\frac{x}{8}}, & 0 < x < \infty, \\
0, & \text{elsewhere}.
\end{cases}
\end{align*}
where T$\brak{\cdot}$ denotes the gamma function. Also ,let $\Bar{X}_{n}=\frac{1}{n}\brak{X_{1}+X_{1}+\cdots+X_{n}}.$ If $\sqrt{n}\brak{\Bar{X}_{n}\brak{3-\Bar{X}_{n}}-\frac{20}{9}}$ converges to V$\brak{0,\sigma^{2}}$ in distribution,then $\sigma^{2} \brak{\text{ (rounded
off to two decimal places }}$ is equal to $\underline{\hspace{2cm}}.$
\bigskip
\item Consider a Poisson process $\cbrak{X\brak{t},t\geq o}$
The probability mass function of X$\brak{t}$ is given by 
\begin{align*}
     f\brak{t}=\frac{e^{-4t}\brak{4t}^{n}}{n!}, n=0,1,2,\cdots
\end{align*}
if C$\brak{t_{1},t_{2}}$ is the covariance function of the Poisson process, then the value of C$\brak{5,3} \brak{\text{ in integer }}$ is equal to $\underline{\hspace{2cm}}.$
\bigskip
\item A random sample of size $4$ is taken from the distribution with the probability density function 
\begin{align*}
    f\brak{x;\theta} = 
\begin{cases} 
\frac{2\brak{\theta-x}}{\theta^{2}}, & 0 < x < \theta, \\
0, & \text{elsewhere}.
\end{cases}
\end{align*}
If the observed sample values are $6, 5, 3, 6,$ then the method of moments estimate $\brak{\text{ in integer }}$ of the parameter $\theta$, based on these observations, is $\underline{\hspace{2cm}}.$
\bigskip
\item A company sometimes stops payments of quarterly dividends. If the company pays the quarterly dividend, the probability that the next one will be paid is $0.7.$ If the
company stops the quarterly dividend, the probability that the next quarterly dividend will not be paid is $0.5.$ Then the probability $\brak{\text{ (rounded
off to two decimal places }}$ that the company will not pay quarterly dividend in the long run is$\underline{\hspace{2cm}}.$
\bigskip
\item Let $X_{1},X_{2},\cdots,X_{8},$ be a random sample taken from a distribution with the probability density function
\begin{align*}
    f_{X}\brak{x} = 
\begin{cases} 
\frac{x}{8}, & 0 < x < 4, \\
0, & \text{elsewhere}.
\end{cases}
\end{align*}
Let $F_{8}\brak{x}$ be the empirical distribution function of the sample. If $\alpha$ is the variance of  $F_{8}\brak{2},$ then $128\alpha$  $\brak{\text{ in integer }}$  is equal to $\underline{\hspace{2cm}}.$
\bigskip
\item Let M be a $3\times3$ real symmetric matrix with eigenvalues $-1,1,2$ and the corresponding unit eigenvectors u,v,w, respectively . Let x and y be two vectors in $\mathbb{R}^{3}$ such that
\begin{align*}
    Mx=u+2\brak{v+w} \text{ and }  M^{2}y=u-\brak{v+2w}
\end{align*}
Considering the usual inner product in $\mathbb{R}^{3},$ the value of $\sbrak{x+y}^{2},\text{ where }\sbrak{x+y}$ is the length of the vector $x+y , $ is 
\begin{enumerate}
    \item $1.25$
    \item $0.25$
    \item $0.75$
    \item $1$
    
\end{enumerate}
\bigskip
\item Consider the following infinite series:
\begin{align*}
    S_{1}:=\sum_{n=0}^{\infty}\brak{-1}^{n}\frac{n}{n^{2}+4} \text{ and } S_{2}:=\sum_{n=0}^{\infty}\brak{-1}^{n}\brak{\sqrt{n^{2}+1}-n}. 
\end{align*}
Which of the above series is/are conditionally convergent?
\begin{enumerate}
    \item $S_{1}$ only
    \item $S_{2}$ only
    \item Both $S_{1} \text{ only } S_{2}$ 
    \item Neither  $S_{1} \text{ nor } S_{2}$ 
\end{enumerate}
\bigskip
\item Let $\brak{3,6}^{T},\brak{4,4}^{T},\brak{5,7}^{T},\text{ and }\brak{3,6}^{T}$ be four independent observations from a
bivariate normal distribution with the mean vector $\mu$ and the covariance matrix $\sum.$ Let $\hat{\mu} \text{ and } \hat{\sum}$ be the maximum likelihood estimates of $\mu \text{ and } \sum,$ respectively, based on
these observations. Then $\hat{\sum}\hat{\mu}$ is equal to
\begin{enumerate}
    \item $\brak{\begin{array}{c} 3.5 \\ 10 \end{array}}$
    \item $\brak{\begin{array}{c} 7.5 \\ 4 \end{array}}$
    \item $\brak{\begin{array}{c} 4 \\ 13.5\end{array}}$
    \item $\brak{\begin{array}{c} 10 \\ 3.5\end{array}}$
\end{enumerate}
\bigskip
\item Let X=$\brak{\begin{array}{c} x_{1} \\ x_{2}\\x_{3} \end{array}}$  follow $N_{3}\brak{\mu,\sum}\text{ with }\mu=\brak{\begin{array}{c} 2 \\ -3 \\ 2 \end{array}}$
and $\sum=\begin{bmatrix} 
4 & -1 & 1 \\ 
-1 & 2 & \alpha \\ 
1 &  \alpha & 2 
\end{bmatrix}$ where $ \alpha\in\mathbb{R}.$ Suppose that the partial correlation coefficient between $X_{2}\text{ and } X_{3}$, keeping $X_{1}$ fixed ,is $\frac{5}{7} \text{ then }\alpha$ is equal to
\begin{enumerate}
    \item $1$
    \item $\frac{3}{2}$
    \item $2$
    \item $\frac{1}{2}$
\end{enumerate}
 


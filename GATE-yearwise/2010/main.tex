\iffalse
\chapter{2010}
\author{AI24BTECH11032}
\section{ma}
\fi

\item Consider the wave equation $\frac{\partial^{2}u}{\partial t^{2}}=4\frac{\partial^{2}u}{\partial x^{2}},0<\text{x}<\pi,\text{t}>0,$ with $\text{u}\brak{0,\text{t}}=\text{u}\brak{\pi,\text{t}}=0,\text{u}\brak{\text{x},0}=\sin x \text{ and }\frac{\partial u}{\partial t}=0 \text{ at } t=0.\text{ Then } u\brak{\frac{\pi}{2},\frac{\pi}{2}}$ is
\begin{multicols}{4}
    \begin{enumerate}
        \item $2$
        \item $1$
        \item $0$
        \item $-1$
    \end{enumerate}
\end{multicols}
\bigskip
\item Let $I=\int_ {C} \frac{e^{\prime}}{x} \, dx + \brak{e^{y}\ln x +x}dy,$ where C is the positively oriented  boundary of the region enclosed by $y=1+x^{2},y=2,x=\frac{1}{2}.$ Then the value of I is 
\begin{multicols}{4}
    \begin{enumerate}
        \item $\frac{1}{8}$
        \item $\frac{5}{24}$
        \item $\frac{7}{24}$
        \item $\frac{3}{8}$
    \end{enumerate}
\end{multicols}
\bigskip
\item Let $\cbrak{f_{n}}$ be a sequence of real valued differentiable functions on $\sbrak{a,b}$ such that int $f_{n}\brak{x}\to f\brak{x} \text{ as } n \to \infty$ for every $x \in \sbrak{a, b}$ and for some Riemann-integrable function $f : \sbrak{a, b} \to R$ Consider the statements
\begin{align*}
    P_{1}:\cbrak{f_{n}} \text{converges uniformly }\\
     P_{2}:\cbrak{f_{n}^{\prime}} \text{converges uniformly }\\
    P_{3}: \int_{n}^{b} f_{n}\brak{x}dx \to \int_{n}^{b} f\brak{x}dx\\
    P_{4}:\text{f is differentiable }
\end{align*}
Then which one of the following need NOT be true
\begin{multicols}{4}
    \begin{enumerate}
        \item $P_{1} \text{ implies } P_{1}$
        \item $P_{2} \text{ implies } P_{1}$
        \item $P_{2} \text{ implies } P_{4}$
        \item $P_{3} \text{ implies } P_{1}$
    \end{enumerate}
\end{multicols}
\bigskip
\item Let $f_n\brak{x}=\frac{x^{n}}{1+x} \text{ and } g_{n}\brak{x}=\frac{x^{n}}{1+nx} \text{ for } x\in \sbrak{0,1} \text{ and } n\in N.$ Then on the interval $\sbrak{0,1}.$

    \begin{enumerate}
        \item both $\cbrak{f_{n}}$ and $\cbrak{g_{n}}$ converge uniformly
        \item  neither  $\cbrak{f_{n}}$  nor $\cbrak{g_{n}}$ converges uniformly
        \item $\cbrak{f_{n}}$ converges uniformly but $\cbrak{g_{n}}$ does not converge uniformly
        \item $\cbrak{g_{n}}$ converges uniformly but $\cbrak{f_{n}}$ does not converge uniformly
    \end{enumerate}
\bigskip
\item consider the power series $\sum_{n=1}^{\infty}\frac{x^{n}}{\sqrt{n}} \text{ and } \sum_{n=1}^{\infty}\frac{x^{n}}{n} .$ Then 

    \begin{enumerate}
        \item both converge on $(-1,1]$
        \item both converge on $[-1,1)$
        \item exactly one of them converges on $(-1,1]$
        \item none of them converges on $[-1,1)$
    \end{enumerate}
\bigskip 
\item Let X=N be equipped with the topology generated by the basis consisting of sets $A_{n}=\brak{n,n+1,n+2\cdots},n\in \text{N}.$ Then X is 
\begin{multicols}{2}
    \begin{enumerate}
        \item Compact and connected 
        \item Hausdorff and connected 
        \item Hausdorff and compact
        \item Neither Compact nor connected
    \end{enumerate}
\end{multicols}
\bigskip
\item Four weightless rods form a rhombus PQRS with smooth hinges at the joints. Another weightless rod joins the midpoints E and F of PQ and PS respectively. The system is suspended from P and a weight 2W is attached to R. If the angle between the rods PQ and PS is $2\theta$. then the thrust in the rod EF is
\begin{multicols}{4}
    \begin{enumerate}
        \item W $\tan\theta$
        \item 2W $\tan\theta$
        \item W $\cot\theta$
        \item 4W $\tan\theta$
    \end{enumerate}
\end{multicols}
\bigskip
\item For a continuous  function f$\brak{t},0\leq r \leq 1$ the integral equation y$\brak{t}=f\brak{t}+3\int_{0}^{1} \text{ts }  y\brak{s}ds$ has 

    \begin{enumerate}
        \item a unique solution if $\int_{0}^{1} \text{s}f\brak{s}ds \neq 0$
        \item no  solution if $\int_{0}^{1} \text{s}f\brak{s}ds=0$
        \item  infinitely many solution if $\int_{0}^{1} \text{s}f\brak{s}ds=0$
        \item infinitely many solution if $\int_{0}^{1} \text{s}f\brak{s}ds \neq 0$
    \end{enumerate}

\bigskip
$$\textbf{Common Data Question}$$
$\textbf{Common Data for  Question 48 and 49:}$\\
Let X and Y be continuous random variables with the joint probability density  function
f$\brak{x, y}$ = 
$\begin{cases} 
    ae^{-zy}, & 0 < x < y < \infty \\ 
    0, & \text{otherwise} 
\end{cases}$
\item The value of a is 
\begin{multicols}{4}
    \begin{enumerate}
        \item $4$
        \item $2$
        \item $0$
        \item $0.5$
    \end{enumerate}
\end{multicols}
\bigskip
\item the value of of E$\brak{X\mid Y=2}\text{is} $
\begin{multicols}{4}
    \begin{enumerate}
        \item $4$
        \item $3$
        \item $2$
        \item $1$
    \end{enumerate}
\end{multicols}
\bigskip
$\textbf{Common Data for  Question 50 and 51:}$\\
Let X=N$\times$Q with the subspace topology of the usual topology on $R^{2}$ and P=$\cbrak{\brak{n,\frac{1}{n}}:n\in \text{N}}.$
\bigskip
\item In the space X, 
\begin{multicols}{2}
    \begin{enumerate}
        \item P is closed but is not open 
        \item P is open but is not closed 
        \item P is both  open and  closed 
        \item P is neither open but nor closed 
    \end{enumerate}
\end{multicols}
\bigskip
\item The boundary of P and X is 
\begin{multicols}{4}
    \begin{enumerate}
        \item an empty set 
        \item a singleton set  
        \item P 
        \item X 
    \end{enumerate}
\end{multicols}
\bigskip
$$\textbf{Linked Answer Question}$$
$\textbf{Statement for linked Answer Questions 52 and 53:}$\\
For a differentiable function f$\brak{x},$ the integral $\int_{0}^{h}f\brak{x}dx$ is approximated by the formula $h\sbrak{a_{0}f\brak{0}+a_{1}f\brak{h}}+h^{2}\sbrak{b_{0}f^{\prime}\brak{0}+b_{1}f^{\prime}\brak{h}},$ which is exact for all polynomials of degree at most 3.\bigskip
\item The value of $a_{1}$ and $b_{1}$ respectively are 
\begin{multicols}{4}
    \begin{enumerate}
        \item $\frac{1}{2} \text{ and } -\frac{1}{12}$
        \item $-\frac{1}{12} \text{ and } \frac{1}{2}$ 
        \item$\frac{1}{2} \text{ and } \frac{1}{12}$
        \item$\frac{1}{12} \text{ and } -\frac{1}{2}$
    \end{enumerate}
\end{multicols}
\bigskip
\item The values of $a_{0}$ and $b_{0}$ respectively  are 
\begin{multicols}{4}
    \begin{enumerate}
        \item $\frac{1}{2} \text{ and } \frac{1}{2}$
        \item $\frac{1}{12} \text{ and } -\frac{1}{12}$ 
        \item$\frac{1}{2} \text{ and } \frac{1}{12}$
        \item$\frac{1}{2} \text{ and } -\frac{1}{12}$
    \end{enumerate}
\end{multicols}




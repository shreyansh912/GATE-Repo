\iffalse
\chapter{2024}
\author{AI24BTECH11032}
\section{ma}
\fi
\item Which of the following is/are eigenvalue(s) of the Sturm-Liouville problem 
\begin{align*}
    y^{n}+\lambda y=0, 0\leq x\leq \pi,\\
    y\brak{0}=y\prime\brak{0}\\
    y\brak{\pi}=y\prime\brak{\pi}?
\end{align*}
\begin{enumerate}
    \item $\lambda=1$
    \item $\lambda=2$
    \item $\lambda=3$
    \item $\lambda=4$
\end{enumerate}
\bigskip
\item Let $f:\mathbb{R}^{2}\to R$ be a function such that 
\begin{align*}
    f\brak{x,y}=
    \begin{cases}
        \brak{1-\cos\frac{x^{2}}{y^{2}}}\sqrt{x^{2}+y^{2},} \text{ If y }\neq 0,x\in\mathbb{R}\\
        0, \text{ otherwise }
    \end{cases}
\end{align*}
Which of the following is/are correct?
\begin{enumerate}
    \item f is continuous at $\brak{0, 0}$, but not differentiable at $\brak{0, 0}$
    \item f is differentiable at $\brak{0, 0}$
    \item All the directional derivatives of f at $\brak{0, 0}$ exist and they are equal to zero
    \item Both the partial derivatives of f at $\brak{0, 0}$ exist and they are equal to zero
\end{enumerate}
\bigskip
\item For an integer n, let $f_{n}\brak{x}=xe^{-nx}\text{ where } x\in\sbrak{0,1}.\text{ Let S }:=\cbrak{f_{n}:n\geq1}.$ Consider the matric space $\brak{C\brak{\sbrak{0,1}},d} $ where 
\begin{align*}
    d\brak{f,g}=\text{sup}_{x\in\brak{0,1}}\cbrak{\abs{f\brak{x}-g\brak{x}}}, f,g\in C\brak{\sbrak{0,1}}.
\end{align*}
Which of the following statement(s) is/are true?
\begin{enumerate}
    \item S is an equi-continuous family of continuous functions
    \item S is closed in $\brak{C\brak{\sbrak{0,1}},d}$
    \item S is bounded in $\brak{C\brak{\sbrak{0,1}},d}$
    \item S is compact in $\brak{C\brak{\sbrak{0,1}},d}$
\end{enumerate}
\bigskip
\item Let $T:\mathbb{R}^{4}\to\mathbb{R}^{4}$ be an $\mathbb{R}$-linear transformation such that $1$ and $2$ are the only eigenvalues of T Suppose the dimensions of Kernel $\brak{\text{ T }-I_{4}}$ and  Range$\brak{T-2 I_{4}}$
are$ 1 $and $2$, respectively. Which of the following is/are possible (upper triangular) Jordan canonical form(s) of T?
\begin{enumerate}
    \item $\begin{pmatrix}
1 & 0 & 0 & 0 \\
0 & 2 & 0 & 0 \\
0 & 0 & 2 & 1 \\
0 & 0 & 0 & 2 \\
\end{pmatrix}$
\item $\begin{pmatrix}
1 & 0 & 0 & 0 \\
0 & 2 & 1 & 0 \\
0 & 0 & 2 & 1 \\
0 & 0 & 0 & 2 \\
\end{pmatrix}$
\item $\begin{pmatrix}
1 & 1 & 0 & 0 \\
0 & 1 & 0 & 0 \\
0 & 0 & 2 & 1 \\
0 & 0 & 0 & 2 \\
\end{pmatrix}$
\item $\begin{pmatrix}
1 & 1 & 0 & 0 \\
0 & 1 & 0 & 0 \\
0 & 0 & 2 & 0 \\
0 & 0 & 0 & 2 \\
\end{pmatrix}$

\end{enumerate}
\bigskip
\item Let $L^{2}\brak{\sbrak{-1,1}}$  denote the space of all real-valued Lebesgue square-integrable functions on $\sbrak{-1,1},$ with the usual norm $\abs{}\cdot\abs{}.$ Let $P_{I}$ be the subspace of $L^{2}\brak{\sbrak{-1,1}}$ consisting of all the polynomials of degree at most $1$. Let
$f\in L^{2}\brak{\sbrak{-1,1}}$ be such that $\abs{\abs{f}}^{2}=\frac{18}{5},\int_{-1}^{1}f\brak{x}dx=2,\text{ and }\int_{-1}^{1}f\brak{x}dx=0.$ Then 
\begin{align*}
    \text{ inf }_{g\in P_{1}}\abs{\abs{f-g}}^{2}= \underline{\hspace{2cm}}
\end{align*}
$\brak{\text{ Round off to TWO decimal places}}$
\bigskip
\item The maximum value of $f\brak{x,y,z}=10x+6y-8z$ subject to the constraints
\begin{align*}
    5x-2y+6z\leq 20\\10x+4y-6z\leq30\\x,y,a\geq,\\
\end{align*}
is equal to $\underline{\hspace{2cm}} \brak{\text{ Round off to TWO decimal places}}$
\bigskip
\item Let K$\subseteq$C be the field extension of $\mathbb{Q}$ Q obtained by adjoining all the roots of the polynomial equation $\brak{X^{2}-2}\brak{X^{3}-3}=0$ The number of distinct fields F such that $\mathbb{Q}\subseteq{F}\subseteq{K}$ is equal to $\underline{\hspace{2cm}} \brak{\text{ answer in integer}}$
\bigskip
\item Let H be the subset of $S_{3}$ consisting of all $\sigma\in S_{3}$ such that
\begin{align*}
    \text {Trace }\brak{A_{1}A_{2}A_{3}}=\text{ Trace }\brak{A_{\sigma\brak{1}}A_{\sigma\brak{2}}A_{\sigma\brak{3}}}
\end{align*}
for all $A_{1},A_{2},A_{3}\in M_{2}\brak{C}.$The number of elements in H is equal to $\underline{\hspace{2cm}} \brak{\text{ answer in integer}}$
\bigskip
\item Let r$:\sbrak{0,1}\to\mathbb{R}^{2}$ be a continuously differentiable path from $\brak{0,2}\text{ to }\brak{3,0}$ and let F$:\mathbb{R}^{2}\to\mathbb{R}^{2}$ be defined F$\brak{x,y}=\brak{1-2y,1-2x}.$ The line integral of F along r 
\begin{align*}
    \int F.dr
\end{align*}
is equal to $\underline{\hspace{2cm}} \brak{\text{ round off to TWO decimal places }}$
\bigskip
\item Let $u=u\brak{x,t}$ be the solution of the initial value problem 
\begin{align*}
    \frac{\partial^{2}u}{\partial t^{2}}-\frac{\partial^{2}u}{\partial x^{2}}=0,x\in\mathbb{R},t>0,\\
    u\brak{x,0}=0,x\in\mathbb{R},\\
    \frac{\partial u}{\partial t}\brak{x,0}=
    \begin{cases}
        x^{4}\brak{1-x}^{4}, \text{ If }0<x<1\\
        0, \text{ otherwise }
    \end{cases}
\end{align*}
If $\alpha=\text{ inf }\cbrak{t>0:u\brak{2,t}>0},$ then $\alpha$ is equal to $\underline{\hspace{2cm}} \brak{\text{ round off to TWO decimal places }}$
\bigskip
\item The boundary value problem
\begin{align}
    x^{2}y\prime\prime=2xy\prime+2y=0, 1\leq x\leq2,\\
    y\brak{1}-y\prime\brak{1}=1,\\
    y\brak{2}=ky\prime\brak{2}=4,\\
\end{align}
has infinitely many distinct solutions when k is equal to   $\underline{\hspace{2cm}} \brak{\text{ round off to TWO decimal places }}$
\bigskip
\item The global maximum of f$\brak{x,y}=\brak{x^{2}+y^{2}e^{2-x-y}}$ on $\cbrak{\brak{x,y}\in\mathbb{R}^{2}:x\geq0,y\geq0}$ is equal to $\underline{\hspace{2cm}} \brak{\text{ round off to TWO decimal places }}$
\bigskip
\item Let $k\in\mathbb{R}$ and D$=\cbrak{\brak{r,\theta}:0<r<2,0<\theta<\pi}.$ Let $u\brak{r,\theta}$ be the solution of the
following boundary value problem
\begin{align*}
    \frac{\partial^{2}u}{\partial r^{2}}+\frac{1}{r}\frac{\partial u}{\partial r}+\frac{1}{r^{2}}\frac{\partial^{2}u}{\partial \theta^{2}}=0,\brak{r,\theta}\in D,\\
    u\brak{r,0}=u\brak{r,\pi}=0, 0\leq r\leq2,\\
    u\brak{2,\theta}=k\sin\brak{2\theta},0<\theta<\pi\\
\end{align*}
If $\text{ u }\brak{1,\frac{\pi}{4}}=2,$ then the value of k is equal to $\underline{\hspace{2cm}} \brak{\text{ round off to TWO decimal places }}$


s

\iffalse
\chapter{2018}
\author{AI24BTECH11032}
\section{me}
\fi
    \item A machine of mass $m= 200 kg$is supported on two mounts, each of stiffness
$k=10\frac{kN}{m}.$ The machine is subjected to an external force $\brak{\text{in N}}$ $F\brak{t}=50\cos5t.$ Assuming only vertical translatory motion, the magnitude of the dynamic force $\brak{\text{in N}}$ transmitted from each mount to the ground is $\underline{\hspace{2cm}} \brak{\text{correct to two decimal places}}.$
\begin{figure}[H]
\centering
\resizebox{0.4\textwidth}{!}{%
\begin{circuitikz}
\tikzstyle{every node}=[font=\large]
\draw [ line width=1.4pt ] (2.75,11.25) rectangle (17.75,1.75);
\draw [ line width=1.4pt](6,1.75) to[R] (6,-3.75);
\draw [ line width=1.4pt](14,1.75) to[R] (14,-3.75);
\draw [ line width=1.4pt](2.75,-3.75) to[short] (18.25,-3.75);
\draw [line width=1.4pt, short] (4.25,-3.75) -- (2.5,-5.25);
\draw [line width=1.4pt, short] (5.5,-3.75) -- (4,-5.25);
\draw [line width=1.4pt, short] (6.75,-3.75) -- (5.25,-5.25);
\draw [line width=1.4pt, short] (8,-3.75) -- (6.5,-5.5);
\draw [line width=1.4pt, short] (9.5,-3.75) -- (8,-5.5);
\draw [line width=1.4pt, short] (11,-3.75) -- (9.25,-5.5);
\draw [line width=1.4pt, short] (12.25,-3.75) -- (10.5,-5.75);
\draw [line width=1.4pt, short] (13.5,-3.75) -- (12,-5.75);
\draw [line width=1.4pt, short] (14.5,-3.75) -- (13,-6);
\draw [line width=1.4pt, short] (15.5,-3.75) -- (14,-6);
\draw [line width=1.4pt, short] (16.5,-3.75) -- (15,-6);
\draw [line width=1.4pt, ->, >=Stealth] (10,5.25) -- (9.75,16.5);
\node [font=\large] at (11,13.25) {F$\brak{t}$};
\node [font=\large] at (11,8.25) {m};
\node [font=\large] at (5.5,-1) {k};
\node [font=\large] at (14.75,-1) {k};
\end{circuitikz}
}%
\end{figure}
\bigskip
\item A slider crank mechanism is shown in the figure. At some instant, the crank angle is $45^{\circ}$ and a force of $40\text{ N }$ is acting towards the left on the slider. The length of the crank is $30\text{ mm }$ and the connecting rod is $70\text{ mm }.$ Ignoring the effect of gravity, friction and inertial forces, the
magnitude of the crankshaft torque $\brak{\text{in Nm}}$ needed to keep the mechanism in equilibrium is $\underline{\hspace{2cm}} \brak{\text{correct to two decimal places}}.$
\begin{figure}[H]
\centering
\resizebox{0.4\textwidth}{!}{%
\begin{circuitikz}
\tikzstyle{every node}=[font=\large]
\draw [ line width=1.4pt](-1.75,0.75) to[short] (13,0.75);
\draw [ line width=1pt](8.75,2.25) to[short] (8.75,0);
\draw [ line width=1pt](10.25,2.25) to[short] (10.25,0);
\draw [line width=1pt, short] (-1,0.5) .. controls (-1.5,1.75) and (0.25,1.75) .. (0,0.5);
\draw [line width=1.5pt, short] (-1.75,0.5) -- (0.75,0.5);
\draw [line width=1.5pt, short] (-1.5,0.5) -- (-2,0);
\draw [line width=1.5pt, short] (-1,0.5) -- (-1.5,-0.25);
\draw [line width=1.5pt, short] (-0.5,0.5) -- (-1,-0.25);
\draw [line width=1.5pt, short] (0,0.5) -- (-0.5,-0.25);
\draw [line width=1.5pt, short] (19.5,-7.5) -- (19.5,-7.75);
\draw [line width=1.5pt, short] (0.25,0.5) -- (0,-0.25);
\draw [ line width=1.5pt](1,2.5) to[short, -o] (4.5,6) ;
\draw [ line width=1.5pt](1,2.5) to[short, -o] (-0.75,0.75) ;
\draw [ line width=1.5pt](4.5,6) to[short, -o] (9.75,0.75) ;
\draw [ line width=1.5pt](8.25,2.25) to[short] (11,2.25);
\draw [ line width=1.5pt](8.25,2.25) to[short] (9.25,3.25);
\draw [ line width=1.5pt](9,2.25) to[short] (10,3.25);
\draw [ line width=1.5pt](10,2.25) to[short] (11,3.25);
\draw [ line width=1.5pt](10.75,2.25) to[short] (11.5,3);
\draw [ line width=1.5pt](8.25,0.25) to[short] (10.75,0.25);
\draw [ line width=1.5pt](8.5,0.25) to[short] (7.75,-0.5);
\draw [ line width=1.5pt](9.25,0.25) to[short] (8.25,-0.75);
\draw [ line width=1.5pt](10,0.25) to[short] (8.75,-1);
\draw [ line width=1.5pt](10.75,0.25) to[short] (9.5,-1);
\draw [line width=1.5pt, ->, >=Stealth] (13.75,1.25) -- (10.5,1.25);
\node [font=\large] at (12.75,2.25) {40 N};
\end{circuitikz}
}%
\end{figure}
\bigskip
\item A sprinkler shown in the figure rotates about its hinge point in a horizontal plane due to water flow discharged through its two exit nozzles.\begin{figure}[H]
\centering
\resizebox{0.5\textwidth}{!}{%
\begin{circuitikz}
\tikzstyle{every node}=[font=\large]
\draw (1.75,9) to[short] (7.75,9);
\draw (1.75,9) to[short] (1.75,9.5);
\draw (7.75,9) to[short] (7.75,9.5);
\draw (1.25,9.75) to[short] (1.25,8.5);
\draw (1.25,8.5) to[short] (8.25,8.5);
\draw (8.25,8.5) to[short] (8.25,9.75);
\draw (4,9.5) to[short] (4,7.75);
\draw [ line width=2pt ] (4,8.75) circle (0.25cm);
\draw [line width=0.6pt, short] (1.5,8.75) -- (1.5,8);
\draw [line width=0.6pt, short] (8,8.75) -- (8,8);
\draw [line width=0.6pt, ->, >=Stealth] (2.25,8.25) -- (1.5,8.25);
\draw [line width=0.6pt, ->, >=Stealth] (3.25,8.25) -- (4,8.25);
\draw [line width=0.6pt, ->, >=Stealth] (6.75,8.25) -- (8,8.25);
\draw [line width=0.6pt, ->, >=Stealth] (5.5,8.25) -- (4,8.25);
\draw [line width=0.6pt, ->, >=Stealth] (3,8.75) -- (2.25,8.75);
\draw [line width=0.6pt, ->, >=Stealth] (5.75,8.75) -- (6.5,8.75);
\draw [line width=0.6pt, ->, >=Stealth] (8,9.5) -- (8,10);
\draw [line width=0.6pt, ->, >=Stealth] (1.5,9.5) -- (1.5,10);
\draw [line width=0.6pt, short] (3.5,8.75) .. controls (3.5,9.75) and (4.5,9.5) .. (4.5,8.75);
\draw [line width=0.6pt, short] (4.25,9) -- (4.5,8.75);
\draw [line width=0.6pt, short] (4.5,8.75) -- (4.75,9);
\node [font=\large] at (2.75,8.25) {10 cm};
\node [font=\large] at (6.25,8.25) {20 cm};
\node [font=\large] at (6.25,9.75) {$\frac{Q}{2}$};
\node [font=\large] at (3,10) {$\frac{Q}{2}$};
\end{circuitikz}=
}%
\end{figure} The total flow rate Q through the sprinkler is $1\frac{litre}{sec}$ and the cross-sectional area of each exit nozzle is $1\text{ cm }^{2}$. Assuming equal flow rate through both arms and a frictionless hinge, the steady state angular speed of rotation $\brak{\text{ in }\frac{rad}{s}}$ of the sprinkler is $\underline{\hspace{2cm}} \brak{\text{correct to two decimal places}}.$
\bigskip
\item A solid block of $2.0\text{ kg }$ mass slides steadily at a velocity V along a vertical wall as shown in the figure below. A thin oil film of thickness $\text{h} = 0.15 \text{ mm } $ provides lubrication between the block and the wall. The surface area of the face of the block in contact with the oil film is $0,04\text{ m }^{2}$ . The velocity distribution within the oil film gap is linear as shown in the figure. Take dynamic viscosity of oil as$7 \times 10^{-3} \, \text{ Pa }-\text{s}$ and acceleration due to gravity as$10 \frac{m}{s^{2}}$
.Neglect weight of the oil. The terminal velocity V$\brak{\text{ in }\frac{ m }{ s }}$  of the block is $\underline{\hspace{2cm}} \brak{\text{correct to two decimal places}}.$

\begin{figure}[H]
\centering
\resizebox{0.3\textwidth}{!}{%
\begin{circuitikz}
\tikzstyle{every node}=[font=\normalsize]
\draw  (2.5,14.75) rectangle (3.5,-2.75);
\draw [short] (2.5,16.5) -- (2.5,15);
\draw [short] (3.5,16.5) -- (3.5,15);
\draw [short] (2.25,14.75) -- (1.25,14);
\draw [short] (2.5,13.5) -- (1.5,12.75);
\draw [short] (2.5,12.5) -- (1.75,11.75);
\draw [short] (2.5,11.25) -- (1.75,10.5);
\draw [short] (2.5,10.25) -- (1.75,9.5);
\draw [short] (2.5,9.25) -- (1.75,8.25);
\draw [short] (2.5,8) -- (1.75,7);
\draw [short] (2.5,6.75) -- (1.75,5.75);
\draw [short] (2.5,5.75) -- (1.75,4.75);
\draw [short] (2.5,4.25) -- (1.75,3.25);
\draw [short] (2.5,3) -- (1.75,2);
\draw [short] (2.5,1.5) -- (1.75,0.5);
\draw [short] (2.5,0.25) -- (2,-0.75);
\draw [short] (2.5,-1.5) -- (2,-2.25);
\draw [->, >=Stealth] (0.5,15.75) -- (2.5,15.75);
\draw [->, >=Stealth] (5.25,15.75) -- (3.5,15.75);
\node [font=\normalsize] at (6.25,16) {h=0.15 mm};
\draw [ line width=1.4pt ] (3.5,9.75) rectangle (9.75,1);
\draw [line width=1.4pt, short] (2.5,6.75) -- (3.5,6.75);
\draw [line width=1.4pt, short] (2.5,6.75) -- (3.5,4.5);
\draw [line width=1.4pt, ->, >=Stealth] (6,6.25) -- (6,4.25);
\draw [line width=1.4pt, ->, >=Stealth] (3,6.75) -- (3,6);
\draw [line width=1.4pt, ->, >=Stealth] (3.25,6.75) -- (3.25,5.5);
\node [font=\normalsize] at (5.75,7.5) {m=0.2 kg};
\node [font=\normalsize] at (6.75,5.25) {V};
\node [font=\normalsize] at (5.25,-0.25) {A=0.04 $m^{2}$};
\node [font=\normalsize] at (2.25,-3.75) {Impermeable };
\node [font=\normalsize] at (2,-4.25) {wall};
\draw [line width=1.4pt, ->, >=Stealth] (1,-3.5) -- (2,3.25);
\end{circuitikz}
}%

\end{figure}

\bigskip
\item A tank of volume  contains a mixture of saturated water and saturated steam at $200^{\circ}$ The mass of the liquid present is 8 kg. The entropy $\brak{\text{ in }\frac{kj}{kg} \text{ k }}$  of the mixture is  $\underline{\hspace{2cm}} \brak{\text{correct to two decimal places}}.$
Property data for saturated steam and water are:
At $200^\circ \text{C}$,
\begin{align*}
p_{\text{sat}} &= 1.5538 \, \text{MPa}, \\
v_f &= 0.001157 \, \frac{m^{3}}{kg}, v_g = 0.12736 \frac{m^{3}}{kg}, \\
s_{fg} &= 4.1014 \, \frac{kJ}{kg}\text{ K }, s_f = 2.3309 \frac{kJ}{kg}\text{ K }.
\end{align*}
\bigskip
\item Steam flows through a nozzle at a mass flow rate of $\dot{m} = 0.1 \frac{\text{ kg }}{\text{ s }} $  with a heat loss of $5 \text{kW}$. The enthalpies at inlet and exit are $2500 \frac{\text{ kJ }}{\text{ kg}}  \text{and} \quad 2350 \frac{\text{ kJ }}{\text{ kg}} $, respectively. Assuming negligible velocity at inlet $\brak{C_{1}\approx 0},\text{ the velocity }\brak{C_{2}}$of steam $\brak{\text{ in }\frac{m}{s}}$  at the $\underline{\hspace{2cm}} \brak{\text{correct to two decimal places}}.$
\begin{figure}[H]
\centering
\resizebox{0.6\textwidth}{!}{%
\begin{circuitikz}
\tikzstyle{every node}=[font=\large]
\draw [line width=1.7pt, short] (-14.5,13) .. controls (-10.75,8.25) and (-5.5,5.75) .. (2,7.75);
\draw [line width=1.7pt, short] (-12.75,-0.25) .. controls (-8.5,6.25) and (-5.5,5.5) .. (2,4.75);
\draw [line width=1.7pt, ->, >=Stealth] (-14,5.75) -- (-9,5.75);
\draw [line width=1.7pt, ->, >=Stealth] (-7.25,8.25) -- (-5.5,11.5);
\node [font=\large] at (-4.25,12) {$\dot m$=5kw};
\node [font=\large] at (-16.5,6.5) {$h_1=2500\frac{kj}{kg}$};
\node [font=\large] at (-17.5,5.5) {$c_1\approx0$};
\node [font=\large] at (-6.75,5.75) {$\dot m=0.1\frac{kg}{s}$};
\node [font=\large] at (3.5,7) {$h_{2}=2350 \frac{kj}{kg}$};
\node [font=\large] at (1.75,5.5) {$C_{2}$};
\end{circuitikz}
}%

\end{figure}

\bigskip
\item  An engine working on air standard Otto cycle is supplied with air at $0.1\text{ MPa and }35^{\circ}$ The compression ratio is $8.$ The heat supplied is$500 \frac{\text{ kJ }}{\text{ kg}}$ Property data for air: $c_{p}=1.005 \frac{\text{ kJ }}{\text{ kg}},c_{v}=0.718\frac{\text{ kJ }}{\text{ kg}},R=0.287\frac{\text{ kJ }}{\text{ kg}}$ K.The maximum temperature $\brak{\text{ in K }}$ of the  cycle is $\underline{\hspace{2cm}} \brak{\text{correct to two decimal places}}.$
\bigskip
\item A plane slab of thickness L and thermal conductivity k is heated with a fluid on one side $\brak{\text{P}}$, and the other side $\brak{\text{Q}}$  is maintained at a constant temperature, $T_{Q} \text{ of } 25^{\circ}\text{C},$ as shown in the
figure. The fluid is at $45^{\circ}\text{C}$ and the surface heat transfer coefficient, h, is $10\frac{\text{ W }}{\text{m}^{2}\text{K}}$ The
steady state temperature, $T_{P}\brak{\text{ in } ^\circ\text{C}}$ of the side which is exposed to the fluid is $\underline{\hspace{2cm}} \brak{\text{correct to two decimal places}}.$
\begin{figure}[H]
\centering
\resizebox{0.5\textwidth}{!}{%
\begin{circuitikz}
\tikzstyle{every node}=[font=\small]
\draw [ line width=0.6pt](2,10.5) to[short] (2,6.5);
\draw [ line width=0.6pt](4,10.5) to[short] (4,6.5);
\draw [line width=0.6pt, short] (2,10.5) .. controls (2.5,11) and (2.5,10.25) .. (2.75,10.5);
\draw [line width=0.6pt, short] (2,6.5) .. controls (2.5,7) and (2.5,6) .. (2.75,6.5);
\draw [line width=0.6pt, short] (2.75,6.5) .. controls (3.25,7) and (3.25,6) .. (3.5,6.5);
\draw [line width=0.6pt, short] (3.5,6.5) .. controls (3.75,7) and (3.75,6) .. (4,6.5);
\draw [line width=0.6pt, short] (2.75,10.5) .. controls (3,11.25) and (3,10) .. (3.5,10.5);
\draw [line width=0.6pt, short] (3.5,10.5) .. controls (4,11.5) and (3.5,10) .. (4,10.5);
\draw [line width=0.6pt, ->, >=Stealth] (3.5,7.5) -- (4,7.5);
\draw [line width=0.6pt, ->, >=Stealth] (2.75,7.5) -- (2,7.5);
\node [font=\small] at (3,7.75) {L=20 cm};
\node [font=\small] at (3,8.75) {k=2.5$\frac{W}{mK}$};
\node [font=\small] at (1,9.5) {h=10$\frac{W}{m^{2}K}$};
\node [font=\small] at (1.75,10.5) {$T_{P}$};
\node [font=\small] at (5,9.75) {$T_{Q}=25^{\circ}\text{C}$};
\node [font=\small] at (1,8.25) {$T=45^{\circ}\text{C}$};
\end{circuitikz}
}%

\end{figure}


\bigskip
\item The true stress $\brak{\sigma}-$ true strain $\brak{\epsilon}$ diagram of a strain hardening material is shown in figure. First, there is loading up to point A,  up to stress of $500\text{ MPa }$ and strain of $0.5$ . Then from point A,there is unloading up to point B,$\text{i}.\text{e}. ,$ to stress of $100\text{ MPa }.$ Given that the Young's modulus $\text{ E }= 200\text{GPa},$ the natural strain at point B$\brak {\epsilon_{\text{B}} }$ is $\underline{\hspace{2cm}} \brak{\text{correct to two decimal places}}.$
\begin{figure}[H]
\centering
\resizebox{0.4\textwidth}{!}{%
\begin{circuitikz}
\tikzstyle{every node}=[font=\large]
\draw [line width=1.7pt, ->, >=Stealth] (-10.5,1) -- (6.25,1);
\draw [line width=1.7pt, ->, >=Stealth] (-10.5,1) -- (-10.75,17);
\draw [line width=1.7pt, dashed] (-10.75,14) -- (3.75,14.25);
\draw [line width=1.7pt, dashed] (3.75,14.25) -- (4.25,1.25);
\draw [line width=1.7pt, dashed] (-10.5,4.75) -- (4,4.75);
\draw [line width=1.7pt] (-10.5,1.25) .. controls (-9,12) and (-4,11) .. (5,14.75);
\draw [line width=1.7pt] (3.75,14.25) -- (0.25,4.75);
\draw [line width=1.7pt] (-6,11) -- (-5,11);
\draw [line width=1.7pt] (-5,11) -- (-5,10.25);
\draw [line width=1.7pt] (1.5,10) -- (1.75,8.75);
\draw [line width=1.7pt] (1.75,8.75) -- (2.75,9.5);
\draw [line width=1.7pt, dashed] (0.25,4.75) -- (0.25,1);
\draw [line width=1.7pt] (-10.5,0.25) -- (-10.5,-1.25);
\draw [line width=1.7pt] (0.25,0.5) -- (0.25,-1.25);
\draw [line width=1.7pt, <->, >=Stealth] (-10.5,-0.5) -- (0.25,-0.25);

\node [font=\large] at (-11.25,14) {500};
\node [font=\large] at (-11.25,4.75) {100};
\node [font=\large] at (-11.75,16.25) {$\sigma$};
\node [font=\large] at (-12,15.25) {$\brak{\text{MPa}}$};
\node [font=\large] at (-5.5,0.25) {$\epsilon_B$};
\node [font=\large] at (6.25,0.5) {$\epsilon$};
\node [font=\large] at (4.25,0.25) {0.5};
\node [font=\large] at (-0.25,5.25) {B};
\node [font=\large] at (3.75,15) {A};
\end{circuitikz}
}%
\end{figure}

\bigskip
\item An orthogonal cutting operation is being carried out in which uncut thickness is $0.010\text{ mm }$ cutting speed is $130\frac{\text{ m }}{\text{ min }},$ rake angle $15^{\circ}$and width of cut is $6\text{ mm }$  It is observed that
the chip thickness is $0.015\text{ mm }$ the cutting force is $60\text{ N }$ and the thrust force is $25\text{ N }.$  The ratio of friction energy to total energy is $\underline{\hspace{2cm}} \brak{\text{correct to two decimal places}}.$ 
\bigskip
\item A bar is compressed to half of its original length. The magnitude of true strain produced in the deformed bar is $\underline{\hspace{2cm}} \brak{\text{correct to two decimal places}}.$
\bigskip
\item The minimum value of $3x+5y$ such that :
\begin{align*}
3x + 5y &\leq 15 \\
4x + 9y &\leq 8 \\
13x + 2y &\leq 2 \\
x \geq 0, y \geq 0
\end{align*}
is$\underline{\hspace{2cm}}$
\bigskip
\item Processing times $\brak{\text{ including setup times }}$  and due dates for six jobs waiting to be processed at a work centre are given in the table. The average tardiness $\brak{\text{ in days }}$ using shortest processing time rule is $\underline{\hspace{2cm}} \brak{\text{correct to two decimal places}}.$ 
\begin{figure}[H]
\centering
\resizebox{0.5\textwidth}{!}{%
\begin{circuitikz}
\tikzstyle{every node}=[font=\normalsize]
\draw  (2.5,11.5) rectangle (20.25,3.75);
\draw (7.5,11.5) to[short] (7.5,3.75);
\draw (15,11.5) to[short] (15,3.75);
\draw (2.5,10.5) to[short] (20.25,10.5);
\draw (2.5,9.5) to[short] (20.25,9.5);
\draw (2.5,8.5) to[short] (20,8.5);
\draw (2.5,7.25) to[short] (20.25,7.25);
\draw (2.5,6) to[short] (20.25,6);
\draw (2.5,5) to[short] (20.25,5);
\node [font=\normalsize] at (4.5,11) {Job};
\node [font=\normalsize] at (11,11) {Processing  time $\brak{days}$};
\node [font=\normalsize] at (17.5,11) {Due date $\brak{days}$};
\node [font=\normalsize] at (4.5,10) {A};
\node [font=\normalsize] at (4.5,9) {B};
\node [font=\normalsize] at (4.5,7.75) {C};
\node [font=\normalsize] at (4.5,6.5) {D};
\node [font=\normalsize] at (4.5,5.5) {E};
\node [font=\normalsize] at (4.5,4.25) {F};
\node [font=\normalsize] at (10.5,10) {3};
\node [font=\normalsize] at (10.5,9) {7};
\node [font=\normalsize] at (10.5,6.75) {9};
\node [font=\normalsize] at (10.5,8) {4};
\node [font=\normalsize] at (10.5,5.5) {5};
\node [font=\normalsize] at (10.5,4.25) {13};
\node [font=\normalsize] at (17,10) {8};
\node [font=\normalsize] at (17,9) {16};
\node [font=\normalsize] at (17,8) {4};
\node [font=\normalsize] at (17,6.75) {18};
\node [font=\normalsize] at (17,5.5) {17};
\node [font=\normalsize] at (17,4.25) {19};
\end{circuitikz}
}%
\end{figure}

